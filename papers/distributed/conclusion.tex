\section{Conclusion}
\stlabel{conclusion}

We showed a successful way of parallellising graph-based state space
generation, using a combination of three tools: \GROOVE, \BLISS and \LTSMIN. A
nontrivial step is the encoding of arbitrary graphs into fixed-sized state
vectors.  We concluded that the resulting system scales well with the number of
cores, and has a surprisingly good memory performance --- so good, in fact,
that it might be worth replacing the current \GROOVE data structures. We also
observed that a further performance gain can probably be made by reimplementing
the functionality of \BLISS in order to take advantage of the structure of
edge-labelled graphs.

An interesting question raised in the course of this work is whether
isomorphism checking is a good idea at all. Omitting the canonical graph
computation would ensure that rules have only local effect on the state
vector, giving rise to nontrivial (in)dependencies between transitions. This
in turn would allow more of the functionality of \LTSMIN to be used, namely
the symbolic storage of states. 
Though there are examples where symmetry
reduction has a huge payoff, the same is true, to an even larger degree, for
symbolic representations. This is a subject for future investigation.

%%% Local Variables: 
%%% mode: latex
%%% TeX-master: "main"
%%% End: 
