\section{Related Work}
\label{section-related-work}

The \NAUTY tool was created by McKay \cite{mckay2008:nauty24} and is based on
an algorithm for computing canonical form described in
\cite{mckay1981:practical}. In \cite{kant2010:canonical}, we provide a
comprehensive description of the original McKay's algorithm.

Ullmann \cite{ullmann1976:algorithm} presented a search tree based algorithm for
finding graph or subgraph isomorphisms between two graphs. Messner \& Bunke
\cite{messmer1999:decision,messmer2000:efficient} made an optimised version for
large graphs. The graph matching algorithms by Cordella et al.
\cite{cordella1999:performance,cordella2001:improved,cordella2004:subgraph} also
aim at isomorphism checking for pairs of graphs. They use heuristics and
efficient data structures that are optimised for matching large graphs.
Foggia, Sansone, and Vento compare four one-to-one isomorphism checking
algorithms to \NAUTY \cite{foggia2001:performance}. In many cases \NAUTY
performs comparable to these algorithms or better but in some cases \NAUTY
performs worse or is unable to find an answer, whereas some of the algorithms
are able to find an answer for all test cases.

Darga et al. \cite{darga2008:faster} have optimised the \NAUTY algorithm
for symmetry detection in large and sparse graphs and implemented this new
algorithm in the tool \SAUCY \cite{darga2008:saucy}, which does not produce a
canonical form. Optimisations of McKay's algorithm for large and sparse graphs
have also been done by Junttila \& Kaski \cite{junttila2007:engineering} in
\BLISS \cite{junttila2008:bliss}.

\NAUTY has been used in model checking of systems specified in B by the tool
ProB \cite{turner2007:symmetry,spermann2008:prob}. The states of the B model are
translated to edge-labelled graphs which are again converted to node-coloured
graphs and compared using \NAUTY. In \cite{turner2007:symmetry} a version of the
\NAUTY algorithm is adapted to work for edge-labelled graphs, but the search
tree pruning optimisations of \NAUTY are left out. On the contrary, in
\cite{spermann2008:prob} a conversion from edge-labelled graphs to node-coloured
graphs is used in combination with the original \NAUTY algorithm.
Experimentation shows that the symmetry reduction results in faster model
checking.

\section{Conclusions}
\label{section-conclusions}

The results of our experiments have shown that, contrary to expectations, the
state-or-the-art isomorphism checking tools \NAUTY and \BLISS do not do better
than our own ad-hoc implementation in \GROOVE. On the other hand, \BLISS does
appear to scale better to larger graphs, at least in the edge node conversion.
More experimentation and profiling has to be carried out to determine if this
is really the case, and if so, to explain the phenomenon: for graphs with
little or no actual symmetry, the computational complexity of the \GROOVE
algorithm should instead be better than that of \BLISS.

As future work we have identified the following actions:
\begin{itemize}\itemsep0pt
\item We need more experiments in particular also using larger graphs, in order
  to improve our understanding of the relation between \GROOVE and \BLISS performance.

\item So far we have only compared graphs pairwise, whereas in the
  context of model checking we are actually interested in finding an isomorphic
  representative in a set of previously generated graphs. Experiments should be
  set up to compare the performance of \GROOVE and \BLISS also in that context.

\item We believe that the performance loss in \BLISS is mainly due to the need
  for converting the graphs. The canonical form algorithm beneath \NAUTY and
  \BLISS, however, can in principle easily be adapted to cope with
  edge-labelled graphs directly. We intend to carry out this re-implementation
  to get the best of both worlds.
\end{itemize}
