\subsection{Conversions}
\label{section-conversion}

In order to use the existing tools for isomorphism checking we need a function
from $\GL$ to $\GC$ that preserves and reflects isomorphism of graphs. In this
section we present two conversions that have these properties: a layered
conversion function $\tau_1$ inspired by \cite{mckay2007:nug24} (and used in
\cite{spermann2008:prob}) and a function $\tau_2$ that converts each edge
into a distinctly coloured node. As an example we use a representation of a
bounded buffer, see Fig.~\ref{fig:example-conversion:edge-labelled}. The results
for both methods are shown in Fig.~\ref{fig:example-conversion-tau1} and
Fig.~\ref{fig:example-conversion-tau2} respectively.

\begin{figure}[tbp]
\centering
\begin{tikzpicture}[>=stealth',shorten >=1pt,auto,semithick]
\tikzstyle{every node}=[node distance = 4cm, bend angle = 30]
\foreach \pos/ \name/ \label/ \color/ \textcolor in {
{(2,0.5)/v_0/Buffer/white/black}, {(2,2.5)/v_1/Cell/white/black},
{(4,0)/v_2/Cell/white/black}, {(0,0)/v_3/Cell/white/black}}
\node[node] [fill=\color,text=\textcolor] (\name) at \pos {\label};
\foreach \source/ \label/ \dest in {v_0/first/v_1}
\path (\source) edge [->,bend right] node[midway,right] {\label} (\dest);
\foreach \source/ \label/ \dest in {v_0/last/v_1, v_1/next/v_2,v_2/next/v_3,
v_3/next/v_1}
\path (\source) edge [->,bend left] node[midway] {\label} (\dest);
\end{tikzpicture}
\caption{An edge-labelled graph $G$ representing a bounded buffer. Node
labels represent self-edges. The  `Buffer' node points to the first and
last occupied cells.}
\label{fig:example-conversion:edge-labelled}
\end{figure}

\begin{figure}[tbp]
\centering
\subfloat[The mapping between labels in the edge labelled graph and colours in
the converted graph,  used by $\tau_1$.]
{\label{fig:example-conversion:colour-layers-mapping}
\begin{tabular}[b]{r c c}
& $\{\text{Buffer}\}$ & $\{\text{Cell}\}$\\
& 0                   & 1                 \\
first $\mapsto 0$ & \framebox{$0,0$} & \framebox{$0,1$}\\
last $\mapsto 1$ & \framebox{$1,0$} & \framebox{$1,1$}\\
next $\mapsto 2$ & \framebox{$2,0$} & \framebox{$2,1$}
\end{tabular}
}
\quad\quad
\subfloat[$\tau_1(G)$.]{\label{fig:example-conversion:colour-layers}
\begin{minipage}{3.0in}
\centering
\begin{tikzpicture}[>=stealth',shorten >=1pt,auto,semithick]
      \tikzstyle{every node}=[node distance = 4cm, bend angle = 30]
      \foreach \pos/ \name/ \color/ \textcolor/ \colorcode in {
        {(2,2)/v_0/white/black/0,0}, {(3.5,2)/v_1/black/white/0,1},
{(5,2)/v_2/black/white/0,1}, {(6.5,2)/v_3/black/white/0,1},
        {(2,1)/v_4/yellow/black/1,0}, {(3.5,1)/v_5/gray/white/1,1},
{(5,1)/v_6/gray/white/1,1}, {(6.5,1)/v_7/gray/white/1,1},
        {(2,0)/v_8/green/black/2,0}, {(3.5,0)/v_9/blue/white/2,1},
{(5,0)/v_{10}/blue/white/2,1}, {(6.5,0)/v_{11}/blue/white/2,1}}
          \node[vertex] [circle split] (\name) at \pos {$\name$ \nodepart{lower}
{\scriptsize$\colorcode$}};
      \foreach \source/ \dest in {v_0/v_1, v_4/v_5, v_9/v_{10}, v_{10}/v_{11},
                                  v_0/v_4, v_1/v_5, v_2/v_6, v_3/v_7,
                                  v_4/v_8, v_5/v_9, v_6/v_{10}, v_7/v_{11}}
          \path (\source) edge [->] (\dest);
      \foreach \source/ \dest in {v_{11}/v_9}
          \path (\source) edge [->, bend left] (\dest);

      \node at (1,2) {first};
      \node at (1,1) {last};
      \node at (1,0) {next};

      \node at (2,3) {$\{\text{Buffer}\}$};
      \node at (3.5,3) {$\{\text{Cell}\}$};
      \node at (5,3) {$\{\text{Cell}\}$};
      \node at (6.5,3) {$\{\text{Cell}\}$};
\end{tikzpicture}
\end{minipage}
}
\caption{The result of conversion $\tau_1$ for the example in
Fig.~\ref{fig:example-conversion:edge-labelled}.}
\label{fig:example-conversion-tau1}
\end{figure}

\subsubsection{%
\texorpdfstring{Layered conversion $\tau_1$.}%
{Layered conversion tau 1}}
%
The idea of the first conversion is to create a layer of copies of nodes for
each distinct edge label. Edges can then be added in the layer that corresponds
to their labels. Layers are distinguished by the colours that are given to the
nodes. Nodes that are a copy of the same original are linked to each other by a
chain of edges, i.e., $v$'s copy in layer $n$ has an edge to $v$'s copy in
layer $n+1$. Moreover, self-edges of nodes are encoded in the node colour and
need not to be represented by edges in the converted graph. This means that the
colour of nodes in the converted graph is based on a combination of
\begin{inparaenum}
\item the set of node labels (labels of self-edges) of the corresponding node
in the original graph, and
\item the edge label that is represented by the layer in which the node is
placed.
\end{inparaenum}
This mapping is shown in
Fig.~\ref{fig:example-conversion:colour-layers-mapping} and the result of the
conversion in Fig.~\ref{fig:example-conversion:colour-layers}.

\begin{definition}[Layered conversion function]
\label{def:conversion-layered}
$\tau_1 : \GL \to \GC$ is a  function that maps each $\langle
V,E\rangle \in \GL$ to a
node-coloured graph $\langle V', E', c \rangle$ defined by
\begin{align*}
V'& = \{ (v,l) \mid v \in V \land l \in L_E \} \\
E'& = \{ \bigl((v_1,l),(v_2,l)\bigr) \in V' \times V' \mid (v_1,l,v_2) \in E \} \\
  & \quad {} \cup \{ \bigl((v,l_1),(v,l_2)\bigr) \in V' \times V' \mid
l_2 \succ_{L_E} l_1 \} \\
c& = \{ (v', (l,L_v)) \mid v'=(v,l) \in V' \}
\end{align*}
where $L_E = \{ l \mid (v_1,l,v_2) \in E \land v_1 \neq v_2 \}$ and $L_v = \{ l
\mid (v,l,v) \in E \}$ for all $v\in V$.
\end{definition}
%
The following proposition states that the conversion preserves and reflects
isomorphism. For the proof see the extended version of this
paper~\cite{kant2010:canonical}.

\begin{proposition}
\label{prop:tau-1-injective}
For all $G,H \in \GL$, $G \cong H \iff \tau_1(G) \cong \tau_1(H)$.
\end{proposition}

\subsubsection{%
\texorpdfstring{Edge to node conversion $\tau_2$.}%
{Label to edge conversion tau 2}}
%
The second conversion creates a node for each edge in the original
graph, with a colour corresponding to the label of the edge. Edges are
added between the original nodes and the edge nodes to capture the structure
of the original graph. For this conversion a mapping is
maintained from edge labels to node colours. One special class of colours is
reserved to designate nodes that represent the nodes in the original graph.
These colours represent a set of node labels, i.e., the labels of the self-edges
of the node in the original graph. See
Fig.~\ref{fig:example-conversion:colour-vertices-mapping} for an example of the
mapping and Fig.~\ref{fig:example-conversion:colour-vertices} for the resulting
converted graph.

\begin{figure}[tbp]
\centering
\subfloat[The mapping between labels in the edge labelled graph and colours in
the converted graph, used by $\tau_2$.]
{\label{fig:example-conversion:colour-vertices-mapping}
    \begin{minipage}{1.5in}
    \centering
    \begin{tabular}[b]{c l l}
      $\{\text{Buffer}\}$  & $\mapsto$ & \framebox{$0$}   \\ % white
      $\{\text{Cell}\}$    & $\mapsto$ & \framebox{$1$}   \\  % blue
      first         & $\mapsto$  & \framebox{$2$}   \\  % grey
      last          & $\mapsto$  & \framebox{$3$}  \\  % yellow
      next          & $\mapsto$  & \framebox{$4$}  \\  % green
    \end{tabular}
    \end{minipage}
  }
  \quad\quad
  \subfloat[$\tau_2(G)$.]{\label{fig:example-conversion:colour-vertices}
    \begin{minipage}{2.5in}
    \centering
    \begin{tikzpicture}[>=stealth',shorten >=1pt,auto,semithick]
      \tikzstyle{every node}=[node distance = 4cm, bend angle = 10]
      \foreach \pos/ \name/ \color/ \textcolor/ \colorcode in {
        {(2,1)/v_0/white/black/0}, {(2,4)/v_1/blue/white/1},
        {(4,0.5)/v_2/blue/white/1}, {(0,0.5)/v_3/blue/white/1},
        {(4,2.5)/v_4/green/black/4},
{(2,-.5)/v_5/green/black/4},{(0,2.5)/v_6/green/black/4},
        {(1.5,2.5)/v_7/gray/white/2}, {(2.5,2.5)/v_8/yellow/black/3}}
          \node[vertex] [circle split] (\name) at \pos {$\name$ \nodepart{lower}
{\scriptsize$\colorcode$}};

      \foreach \source/ \dest in {v_1/v_4, v_4/v_2, v_2/v_5, v_5/v_3, v_3/v_6,
v_6/v_1}
          \path (\source) edge [->] (\dest);
      \foreach \source/ \dest in {v_0/v_7, v_7/v_1}
          \path (\source) edge [->] (\dest);
      \foreach \source/ \dest in {v_0/v_8, v_8/v_1}
          \path (\source) edge [->] (\dest);
    \end{tikzpicture}
    \end{minipage}
  }
\caption{The result of conversion $\tau_2$ for the example in
Fig.~\ref{fig:example-conversion:edge-labelled}.}
\label{fig:example-conversion-tau2}
\end{figure}

\begin{definition}[Edge node conversion function $\tau_2$]
\label{def:conversion-labelvertex}
$\tau_2 :\GL \rightarrow \GC$ is a function that maps each $\langle
V,E\rangle \in \GL$ to a node-coloured graph $\langle V', E', c \rangle$ where
\begin{align*}
V'& = V \cup E \\
E'& = \bigl\{ \bigl(v_1,(v_1,l,v_2)\bigr) \mid (v_1,l,v_2) \in E \bigr\}
   \cup \bigl\{ \bigl((v_1,l,v_2),v_2\bigr) \mid (v_1,l,v_2) \in E \bigr\} \\
 c& = \bigl\{ \bigl(v, (\mathsf{v},L_v)\bigr) \mid v \in V \bigr\}
    \cup \bigl\{ \bigl(e, (\mathsf{e},l)\bigr) \mid e=(v_1,l,v_2) \in E, v_1 \neq v_2 \bigr\}
\end{align*}
where $L_v = \{ l \mid (v,l,v) \in E \}$ as before.
\end{definition}
%
Here, $\mathsf{v}$ is used to mark a colour as belonging to a node in the
original graph and $\mathsf{e}$ to mark a colour as representing the label of an
edge in the original graph. This is necessary to distinguish `edge' nodes from
`node' nodes in the resulting graph. Again, the proof of the following
proposition is given in~\cite{kant2010:canonical}.

\begin{proposition}
\label{prop:tau-2-injective}
For all $G,H \in \GL$, $G \cong H \iff \tau_2(G) \cong \tau_2(H)$.
\end{proposition}

\subsubsection{Complexity.}
%
Obviously, the size of the graph is a major factor in checking isomorphism.
For $G=\langle V,E \rangle \in \GL$, with node count $n = |V|$, non-self-edge
count $m = |E \setminus E_S|$ ($E_S$ being the set of self-edges of $G$), and
non-self-edge label count $k = |L_E|$, we have:
%
\begin{inparaenum}
\item[(i)] $\tau_1(G)$ has $n \cdot k$ nodes and ${n \cdot (k - 1)} + m$ edges;
\item[(ii)] $\tau_2(G)$ has $n + m$ nodes and $2 \cdot m$ edges.
\end{inparaenum}
%
Thus, for sparse graphs or graphs with many distinct labels ($m$ small or $k$
large) we expect $\tau_2$ to result in smaller graphs than $\tau_1$.
