\section{Introduction}

In graph-based model checking, where systems are modelled as graph
transformation systems (see \cite{rensink2008:explicit}) and states are graphs,
we are proposing to collapse isomorphic graphs to reduce the state space (see
\cite{rensink2009:isomorphism}). Graph isomorphism checking is not known either
to be in P or to be NP-complete, but is believed to be neither
\cite{messmer1999:decision,weisstein2009:isomorphic}. However, despite its
complexity, experiments have shown that state space reduction modulo isomorphism
can pay off in a model with sufficient symmetry: see \cite{crouzen2008:gossiping}.

The graph isomorphism problem is usually posed as: are two given graphs
isomorphic? This problem is studied mostly for unlabelled, sometimes undirected
graphs (that is, the edges have no labels). In the context of our work,
however, the following conditions apply:
%
\begin{itemize}\itemsep0pt
\item The problem is a more general one: given a graph and a (large) set of
  graphs, does the latter contain a graph isomorphic to the first?

\item The graphs are directed and edge-labelled.

\item The graphs themselves are typically not large (up to a thousand nodes, but
  usually much smaller).
\end{itemize}
%
The first point implies that some algorithms for isomorphism checking, namely
those that work purely on the basis of a pairwise comparison, are unsuitable
for our purpose. The second point implies that the existing state-of-the-art
tools, in particular \NAUTY \cite{mckay2008:nauty24,mckay1981:practical} and
\BLISS \cite{junttila2008:bliss,junttila2007:engineering}, are only usable after
a conversion from labelled to unlabelled graphs. The third point implies that
there is less chance of encountering truly hard cases.

We have implemented isomorphism checking in our tool \GROOVE
\cite{rensink2009:groove}, inspired by the ideas of McKay
\cite{mckay1981:practical} that underlie \NAUTY and \BLISS. (More details are
given below.) A natural question is whether it would be better to use these
existing tools instead, which are dedicated to the task and should therefore be
expected to perform better. Arguing against this is the graph conversion
necessary to use the existing tools. This conversion has two potentially
deleterious effects:
%
% \begin{itemize}\itemsep0pt
\begin{inparaenum}
\item the graph size increases;
\item the structure provided by the edge labels is ``flattened out''.
% \end{itemize}
\end{inparaenum}
%
We report a series of experiments showing that on some graphs typically
encountered in graph-based model checking, in fact our own implementation
outperforms the state-of-the-art tools, even without taking the conversion time
itself into account.

\medskip\noindent The remainder of the paper is structured as follows. In the
next section, we describe two different ways to convert labelled to unlabelled
graphs. Section~\ref{section-experiments} describes the experimental setup and
the results. In Section~\ref{section-related-work} we describe related work,
and in Section~\ref{section-conclusions} we present the final conclusions of
the paper.
