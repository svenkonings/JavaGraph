% -----------------------------------------------------------------------------
% 7-conclusion
% -----------------------------------------------------------------------------

\section{Conclusions and Future Work}
\stlabel{conclusion}

Summarising, the highlights of the approach described in this paper are:
\begin{itemize}
\item \Prolog is tightly integrated with graph-based state space exploration;
\item Queries can uniformly combine static and dynamic aspects of graphs;
\item The framework supports user-defined \Prolog facts and predicates.
\end{itemize}
We have demonstrated these advantages by applying the approach in the domain of
feature modelling, where it gives rise to a competitive alternative to existing,
more rigid frameworks.

We have implemented the above as an extension to \GROOVE. Although many of
the examples given in this paper could have been solved in \GROOVE using other
means, the \Prolog-based solutions are more convenient and elegant. Therefore,
the extension improves usability, which is a key factor for success.

On a more general level, this paper shows that there is much to be gained when
graph transformation is connected to other techniques, and that this connection
can be done in a simple, uniform way.

\paragraph{Future Work.}
There are two main points planned as future work.
%
\begin{itemize}\noitemsep
\item \textit{\Prolog-based application conditions.} One can associate
  \Prolog queries to individual GT rules, to play the role of additional
  application conditions. When a rule with a query is matched, the query is
  executed in the \Prolog interpreter, and only if the query succeeds the rule
  is applied. This functionality is orthogonal to other application conditions
  already present in \GROOVE, such as NACs, and would give another option for
  controlling the flow of rule applications, in addition to rule priorities and
  control programs.
%
\item \textit{\Prolog-based state space exploration.} One can also extend the
  \GROOVE exploration strategies with a condition based on a \Prolog query.
  Every time a new state is produced, the query is run, and if the query is
  successful the state is added to the GTS. The effect is comparable to a
  global post-application condition.
\end{itemize}

\paragraph{Availability.}
The \Prolog extension described in this paper is implemented in \GROOVE version
4.4.0, available at \url{http://groove.cs.utwente.nl}. The grammar for the
solution given in \stref{problem} can also be downloaded at the same address.

\paragraph{Acknowledgement.} The integration of \Prolog into
\GROOVE is originally due to Michiel Hendriks.
%
%%% Local Variables:
%%% mode: latex
%%% TeX-master: "main"
%%% End:
%