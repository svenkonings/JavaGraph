% -----------------------------------------------------------------------------
% title.tex
% -----------------------------------------------------------------------------

\title{Solving the $N$-Queens Problem with \GROOVE\ --\texorpdfstring{\\}{}
Towards a Compendium of Best Practices}%
\short{Solving the $N$-Queens Problem with \GROOVE}
%
\author{Eduardo Zambon\autref{1} and Arend Rensink\autref{2}}%
\institute{%
\autlabel{1}\email{zambon@inf.ufes.br}\\%
Department of Computer Science and Electronics (DCEL/CEUNES)\\%
Federal University of Espirito Santo (UFES), Brazil\par
\autlabel{2}\email{arend.rensink@utwente.nl}\\%
Formal Methods and Tools Group\\%
Department of Computer Science\\%
University of Twente, The Netherlands}%
%
\abstract{%
We present a detailed solution to the $N$-queens puzzle using \GROOVE, a graph
transformation tool especially designed for state space exploration and
analysis. While \GROOVE has been freely available for more than a decade and
has attracted a reasonable number of users, it is safe to say that only a few
of these users fully exploit the tool features. To improve this situation,
using the $N$-queens puzzle as a case study, in this paper we provide an
in-depth discussion about problem solving with \GROOVE, at the same time
highlighting some of the tool's more advanced features. This leads to a list of
best-practice guidelines, which we believe to be useful to new and expert users
alike.
}
%
\keywords{\GROOVE, $N$-Queens Problem, Tool Usage Guidelines}
\maketitle
