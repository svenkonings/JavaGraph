\clearpage
\section{I/O}
\seclabel{io}

Graph grammars are stored in directories with extension ``\textsf{.gps}'' (for
\emph{g}raph \emph{p}roduction \emph{s}ystem). Each such directory contains all
information about one graph grammar, including rules, start state(s), control
program (if any) and system properties, in separate files.

\subsection{Graphs and rules}

The input/output format used by \GROOVE to store rules and graphs is GXL,
which is an XML format for interchange of graphs. See
\url{http://www.gupro.de/GXL/} for information about the format, including its
DTD and XSD; see also \cite{GXL}. There are some conventions in the way we have
used GXL to encode \GROOVE-specific information.

\paragraph{Edge labels.}

Edge labels are encoded as GXL edge attributes of type \textsf{String}. What is
actually stored is the full label, including prefixes, as seen in the Edit
view. Graphs and rules are stored in precisely the same fashion, except that
rules receive the extension ``\textsf{.gpr}'' (for \emph{g}raph
\emph{p}roduction \emph{r}ule) whereas graphs have extension ``\textsf{.gst}''
(for \emph{g}raph \emph{st}ate).

\paragraph{Graph attributes.}

We are using the GXL graph attributes to store some additional information
about the graphs, such as the name, the fact whether it is a rule, a graph or a
type graph, and the priority and enabledness (if it is a rule). For the latter
two see \secref{rule-properties}.

\paragraph{Graph layout.}

Layout encoding is ad hoc, and in fact based on the way the
graph-rendering library, \JGraph, stores this information internally (see
\cite{JGraph}). The information is stored in GXL node and edge attributes.

In previous versions of \GROOVE, layout information was stored in separate
files with a name derived from that of the GXL file by adding the extension
``\textsf{.gl}''. For instance, the layout of the file \textsf{rule.gpr} was
contained in \textsf{rule.gpr.gl}.

\subsection{Control programs}
\seclabel{io-control}

Control programs are stored in plain text in files with extension
``\textsf{.gcp}'' (for \emph{g}raph \emph{c}ontorl \emph{p}rogram) within the
grammar directory. If there is a file \textsf{control.gcp}, this is taken to be
the default control program; others can be loaded and set in the Simulator or
through the system properties (see below).

\subsection{System properties}
\seclabel{io-system-properties}

System properties are stored in a file \textsf{system.properties} within the
grammar directory. The file is a plain text file, with lines of the form
``\textsf{resource = value}'' for the different properties discussed in
\secref{system-properties}.
