\begin{table}[t]
\begin{minipage}{\textwidth}
\begin{center}
\begin{tabular}{|l|c|l|l|}
\hline\hline
\bf Prefix
 & \bf Where?\footnote{Abbreviations:
on \textsf{N}ode / \textsf{E}dge, resp.\ in \textsf{H}ost
graph / \textsf{R}ule / \textsf{T}ype graph}
 & \bf Explanation
 & \bf Sample\footnote{Name of a sample rule system in which this is used, see
 \url{http://sf.net/projects/groove}, samples download} \\
\hline
\remP & \sf NE,HRT
 & Remark node or edge; used for documentation purposes
 & \textsf{attributed-graphs} \\
\hline
\useP & \sf NE,R
 & Declares a node or edge to be a reader (the default value)
 & \\
\delP & \sf NE,R
 & Declares a node or edge to be an eraser
 & \\
\newP & \sf NE,R
 & Declares a node or edge to be a creator
 & \\
\cnewP & \sf NE,R
 & Declares a node or edge to be a conditional creator
 & \\
\notP & \sf NE,R
 & Declares a node or edge to be an embargo
 & \\
\hline
\boolP & \sf NE,HRT
 & On nodes, a boolean value or type; on edges, a boolean operator
 & \\
\intP & \sf NE,HRT
 & On nodes, an integer value or type; on edges, an integer operator
 & \\
\realP & \sf NE,HRT
 & On nodes, a real value or type; on edges, a real operator
 & \\
\stringP & \sf NE,HRT
 & On nodes, a string value or type; on edges, a string operator
 & \\
\argP & \sf E,R
 & Argument edge, from a product node to an attribute value
 & \textsf{attributed-graphs} \\
\prodP & \sf N,R
 & Product node, collecting arguments for an attribute operation
 & \textsf{attributed-graphs} \\
\letP & \sf N,HR
 & Syntactic sugar for attribute assignment
 & \textsf{attributed-graphs} \\
\testP & \sf N,R
 & Attribute condition that must be satisfied for a rule to apply
 & \textsf{attributed-graphs} \\
\hline
\parP & \sf N,R
 & Anonymous or numbered rule parameter node
 & \textsf{parameters} \\
\parinP & \sf N,R
 & Numbered rule input parameter node
 & \\
\paroutP & \sf N,R
 & Numbered rule output parameter node
 & \\
\hline
\absP & \sf NE,T
 & Abstract type node or edge
 & \\
\subP & \sf E,T
 & Inheritance edge between node types
 & \textsf{inheritance} \\
\inP & \sf E,T &
Incoming edge multiplicity declaration
 & \\
\outP & \sf E,T &
Outgoing edge multiplicity declaration
 & \\
\partP & \sf E,T &
Composite edge declaration
 & \\
\importP & \sf N,T &
Indicates that the node is imported from another type graph
 & \\
\hline
\forallP & \sf N,R
 & Universal quantifier node
 & \\
\forallxP & \sf N,R
 & Non-vacuous universal quantifier node
 & \\
\existsP & \sf N,R
 & Existential quantifier node
 & \\
\existsxP & \sf N,R
 & Optional existential quantifier node
 & \\
\nestedP & \sf E,R
 & Quantifier nesting edge
 & \\
\hline
\idP & \sf N,R &
User-defined node identity
 & \\
\hline
\colorP
 & \sf N,RT
 & Defines the text and outline colour of a node or node type
 & \textsf{colours} \\
\edgeP & \sf N,T
 & Defines a node type to be a nodified edge
 & \textsf{bridge} \\
\hline
\pathP & \sf E,R &
Declares the remainder of the text to be a regular expression label
 & \\
\litP & \sf NE,HR &
Declares the remainder of the text to be a literal edge label
 & \\
\typeP & \sf NE,HRT &
Declares the remainder of the text to be a node type
 & \\
\flagP & \sf NE,HRT &
Declares the remainder of the text to be a flag (= node label)
 & \\
\hline\hline
\end{tabular}
\end{center}
\end{minipage}
\caption{Overview of available edit prefixes. (See also the syntax help in the
\GROOVE Simulator).}
\tablabel{prefixes}
\end{table}


%%% Local Variables: 
%%% mode: latex
%%% TeX-master: "usermanual"
%%% End: 
