\chapter{The GROOVE Tool Set}
\chlabel{The GROOVE Tool Set}

\section{Tool Chain}

The GROOVE Tool Set basically consists of the following components:

\begin{itemize}
  \item{the simulator,}
  \item{the editor,}
  \item{the generator,}
  \item{the imager, and}
  \item{the model checker.}
\end{itemize}

\fref{tool-chain} gives an overview of how most of these tools work together, what they expect as input, and what they provide as output. The squares represent those parts of the tools that have been implemented (white background) and are planned in the future (gray background). The elipses represent the inputs and outputs for the different components.

In the following paragraphs we will shortly discuss all the implemented components, i.e. what they require as input and what they provide as output.

\begin{figure}
  \centering
  \includegraphics[scale=0.7]{\figdir/chain}
  \caption{Chain overview of the \GROOVE Tool Set.}
  \flabel{tool-chain}
\end{figure}

\section{The GROOVE Simulator}

The \GROOVE Simulator is a graphical interface enabling the user to control the \GROOVE transformation engine. It provides a direct graphical view on the state graphs, the transformation rules, and the part of the state space explored so far. Furthermore, from the Simulator the user is able to perform various actions on transformation rules, such as editing, disabling, removing, and creating new ones.

\paragraph{Input.} When starting the Simulator, it can be provided from a path to some directory containing a graph production system. Such directories should have the \texttt{.gps} `extension', otherwise the Simulator will not be able to load it succesfully. These graph production systems typically contain a (possibly empty) set of transformation rules together with a start graph named \texttt{start.gst}. When starting the Simulator from the command-line, it is possible to provide it from a different start graph, which may not be in the same directory as the transformation rules are in.

\paragraph{Output.} Apart from the visual feedback, the Simulator does not automatically provide physical output in the sense of log-files, for instance. At any point during the execution of the production system, it is possible to export the state graphs or individual rules or the transition system generated so far, to a GXL-file.


\section{The GROOVE Editor}

The GROOVE Editor provides means for specifying graphs and transformation rules. From the Editor one is able to create new or modify existing graphs and rules. It does not require any input. Its output consists of the graphs and transformation rules saved by the user.


\section{The GROOVE Generator}

The GROOVE Generator is a useful tool when users are only interested in the outcome of applying the rules of a production system to a specific graph as long as possible, without the need to look at the transformation process itself.

For example, when providing the Generator from a graph production system, it can be asked to save all the final states, i.e. the states in which no rule is applicable anymore.

\paragraph{Input.} The required input is the same as for the Simulator, i.e. a path to the directory containing a graph production system. The Generator can also be provided from a different start graph. Furthermore, the Generator has some options for directing the transformation process. These are, among others, the strategy to use for the transformation process and a file to write the results to.

\paragraph{Output.} When the Generator has finished, it should have written the required result to the specified location. Optionally, it creates a log-file containing some statistics about the transformation process.


\section{The GROOVE Imager}

With the GROOVE Imager one is able to create images from both graphs and transformation rules in various formats, such as JPG, PNG, and EPS.

\section{The GROOVE Model Checker}

Currently, the GROOVE Model Checker supports the verification of properties specified as CTL formulae over finite state graph production systems.

The Model Checker is a command-line tool which, given a graph production system generating a finite state space, first does the state space generation and subsequently asks the user for properties to be verified over this state space. The user is able to check multiple properties in sequence without the need to generate the state space over and over again.

\paragraph{Input.} The input for the Model Checker is equivalent to that of the Generator. After the transformation process, the user is asked to enter CTL formulae that have to be checked, or quit the program.

\paragraph{Output.} Currently, the Model Checker only provides output stating that a given temporal formula is satisfied by the system as specified by the graph production system or is not satisfied.